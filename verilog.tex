\documentclass[10pt,twocolumn]{article}
\title{Summary of Key Verilog Features (IEEE 1364) \thanks{Written by Zafar M. Takhirov and Schuyler Eldridge, ICSG@BU}}
%\author{Zafar. M. Takhirov \and Schuyler Eldridge}
%\date{17 September 2012}
\date{\vspace{-5ex}}
\usepackage{listings}

%\usepackage{fancyhdr}

\usepackage[margin=0.3in,landscape]{geometry}
%
%%%%%%%%%%%%%%%%%%%%%%%%%%%%%%%%%%%%%%%%%%%%%%%%%%
\begin{document}
\setlength{\columnsep}{0.5in}
\setlength{\columnseprule}{0.5pt}
\maketitle
%
%%%%%%%%%%%%%%%%%%%%%%%%%%%%%%%%%%%%%%%%%%%%%%%%%%
\lstset{
        language=Verilog,
        %numbers=left, 
        %numberstyle=\tiny, 
        stepnumber=1, 
        numbersep=5pt,
        basicstyle=\ttfamily,
        %framexleftmargin=5mm, 
        frame=single,}  
%       
%%%%%%%%%%%%%%%%%%%%%%%%%%%%%%%%%%%%%%%%%%%%%%%%%%
\section*{Module}\vspace{-10pt}
Encapsulates functionality; may be nested to any depth
\begin{lstlisting}
module module_name (list of ports);
        // Declarations
        Port modes: input, output, inout identifier;
        Nets (e.g., wire A[3:0];)
        Register variable (e.g., reg B[31:0];)
        Constants: (e.g., parameter size=8;)
        Named events
        Continuous assignments (e.g. assign sum = A + B;)
        Behaviors always (cyclic), initial (single-pass)
        specify ... endspecify
        function ... endfunction
        task ... endtask
        // Instantiations
        primitives
        modules
endmodule
\end{lstlisting}
%
%%%%%%%%%%%%%%%%%%%%%%%%%%%%%%%%%%%%%%%%%%%%%%%%%%
\section*{Multi-Input Primitives}\vspace{-10pt}
\begin{lstlisting}
and (out, in1, in2, ..., inN);
nand (out, in1, in2, ..., inN);
or (out, in1, in2, ..., inN);
nor (out, in1, in2, ..., inN);
xor (out, in1, in2, ..., inN);
xnor (out, in1, in2, ..., inN);
\end{lstlisting}
%
%%%%%%%%%%%%%%%%%%%%%%%%%%%%%%%%%%%%%%%%%%%%%%%%%%
\section*{Three-State  Multioutput Primitives}\vspace{-10pt}
\begin{lstlisting}
buf (out1, out2, ..., outN, in);        // buffer
not (out1, out2, ..., outN, in);        // inverter
\end{lstlisting}
%
%%%%%%%%%%%%%%%%%%%%%%%%%%%%%%%%%%%%%%%%%%%%%%%%%%
\section*{Three-State  Multioutput Primitives}\vspace{-10pt}
\begin{lstlisting}
bufif0 (out, in, control); bufif1 (out, in, control);
notif0 (out, in, control); notif1 (out, in, control);
\end{lstlisting}
%
%%%%%%%%%%%%%%%%%%%%%%%%%%%%%%%%%%%%%%%%%%%%%%%%%%
\section*{Pullups and Pulldowns}\vspace{-10pt}
\begin{lstlisting}
pullup (out_y); pulldown (out_y);
\end{lstlisting}
%
%%%%%%%%%%%%%%%%%%%%%%%%%%%%%%%%%%%%%%%%%%%%%%%%%%
\section*{Propagation Delays}\vspace{-10pt}
\begin{lstlisting}
Single delay:           and #3 G1 (y, a, b, c);
Rise/fall:              and #(3,6) G2 (y, a, b, c);
Rise/fall/turnoff:      bufif0 #(3,6,5) (y, x_in, en);
Min:typ:Max:            bufif1 #(3:4:5, 4:5:6, 7:8:9)
                                (y, x_in, en);
\end{lstlisting}
Command line options for single delay value simulation:
\textbf{+maxdelays, +typdelays, +mindelays}
%
%%%%%%%%%%%%%%%%%%%%%%%%%%%%%%%%%%%%%%%%%%%%%%%%%%
\section*{Concurrent Behavioral Statements}\vspace{-10pt}
May execute a level-sensitive assignment of value to a net (keyword:~\textbf{assign}), or may execute the statements of a cyclic (keyword:~\textbf{always}) or single-pass (keyword:\textbf{initial}) behavior. The statements execute sequentially, subject to level-sensitive or edge-sensitive event control expressions.
\subsubsection*{Syntax}\vspace{-10pt}
\begin{lstlisting}
assign net_name = [expression];
always begin [procedural statements] end
initial begin [precedural statements] end
\end{lstlisting}
Cyclic(\textbf{always}) and single-pass (\textbf{initial}) behaviors may be level sensitive and/or edge sensitive
\subsubsection*{Edge sensitive}\vspace{-10pt}
\begin{lstlisting}
always @ (posedge clock)
        q <= data;
\end{lstlisting}
\subsubsection*{Level sensitive}\vspace{-10pt}
\begin{lstlisting}
always @ (enable or data)
        q <= data;
\end{lstlisting}
%
%%%%%%%%%%%%%%%%%%%%%%%%%%%%%%%%%%%%%%%%%%%%%%%%%%
\section*{Data Types: Nets and Registers}\vspace{-10pt}
\textbf{Nets:} Establish structural connectivity between instantiated primitives and/or modules; may be target of a continuous assignment; e.g., \textbf{wire, tri, wand, wor.} \\
Value is determined during simulation by the driver of the net; e.g., a primitive or a continuous assignment. \\
Example: 
\begin{lstlisting}
wire Y = A + B;
\end{lstlisting}
\textbf{Registers:} Store information and retain value until reassigned.\\
Value is determined by an assignment made by a procedural statement.\\
Value is retained until a new assignment is made; e.g., \textbf{reg, integer, real, realtime, time.}\\
Example:
\begin{lstlisting}
always @ (posege clock)
        if (reset) q_out <= 0;
                else q_out <= data_in;
\end{lstlisting}
%
%%%%%%%%%%%%%%%%%%%%%%%%%%%%%%%%%%%%%%%%%%%%%%%%%%
\section*{Procedural Statements}\vspace{-10pt}
Describe logic abstractly; statements execute sequentially to assign value to variables.
\begin{lstlisting}
if (expression_is_true) statement_1; else statement_2;
case (case_expression)
        case_item_0: statement_0;
        case_item_1: begin
                statement_1_0;
                statement_1_1;
        end
        default: statement;
endcase
for (conditions) statement;
repeat constant_expression statement;
while (expression_is_true) statement;
forever statement;
fork statements join //execute in parallel
\end{lstlisting}
%
%%%%%%%%%%%%%%%%%%%%%%%%%%%%%%%%%%%%%%%%%%%%%%%%%%
\section*{Assignments}\vspace{-10pt}
\textbf{Continuous:} Continuously assigns the value of an expression to a net.\\
\textbf{Procedural (Blocking):} Uses the = operator; executes statements sequentially; a statement cannot execute until the preceding statement completes execution. Value is assigned immediately.\\
\textbf{Procedural (Nonblocking):} Uses the <= operator; executes statements concurrently, independent of the order in which they are listed. Values are assigned concurrently.\\
\textbf{Procedural (Continuous):} \\
\textbf{assign ... deassign} overrides procedural assignments to a net. \\
\textbf{force ... release} overrides all other assignments to a net or a register
%
%%%%%%%%%%%%%%%%%%%%%%%%%%%%%%%%%%%%%%%%%%%%%%%%%%
\section*{Operators}\vspace{-10pt}
\begin{center}
\begin{tabular}{ll|ll}
\{\}~\{\{\}\}   & concatenation      & $\mid$                   & bitwise or \\
+~-~*~/         & arithmetic         & \^{}                     & bitwise exclusive-or \\
\%              & modulus            & \^{}$\sim$ or $\sim$\^{} & bitwise equivalence \\
$> \geq < \leq$ & relational         & \&                       & reduction and \\
!               & logical negation   & ~\&                      & reduction nand \\
\&\&            & logical and        & $\mid$                   & reduction or \\
$\mid\mid$      & logical or         & $\sim\mid$               & reduction nor \\
==              & logical equality   & \^{}                     & reduction exclusive-or \\
!=              & logical inequality & \^{}$\sim$ or $\sim$\^{} & reduction exclusive-nor \\
===             & case equality      & $\ll$                    & left shift \\
!==             & case inequality    & $\gg$                    & right shift \\
$\sim$          & bitwise negation   & ?:                       & conditional \\
\&              & bitwise and        & or                       & event or
\end{tabular}
\end{center}
%
%%%%%%%%%%%%%%%%%%%%%%%%%%%%%%%%%%%%%%%%%%%%%%%%%%
\section*{Specify Block}\vspace{-10pt}
\subsection*{Example: Module Path Delays}\vspace{-10pt}
\begin{lstlisting}
specify
        // specparam declarations (min:typ:max)
        specparam t_r = 3:4:5, t_f = 4:5:6;
        ((A,B) *> Y) = (t_r, t_f);      // full
        (Bus_1 => Bus_1) = (t_r, t_f);  // parallel
        if (state == S0) (a,b *> y) = 2;        // state dep
        (posedge clk => (y -: d_in)) = (3,4);   // edge
endspecify
\end{lstlisting}
\subsection*{Example: Timing Checks}\vspace{-10pt}
\begin{lstlisting}
specify
        specparam t_setup = 3:4:5, t_hold = 4:5:6;
        $setup (data, posedge clock, t_setup);
        $hold (posedge clock, data, t_hold);
endspecify
\end{lstlisting}
%
%%%%%%%%%%%%%%%%%%%%%%%%%%%%%%%%%%%%%%%%%%%%%%%%%%
\section*{Memory}\vspace{-10pt}
Declares an array of words.
\subsection*{Example: Memory declaration and readout}\vspace{-10pt}
\begin{lstlisting}
module memory_read_display();
        reg[31:0] mem_array[1:1024];
        integer k;
        initial begin
                // read contents of mem_array from 
                // a file in hex format
                $readmemh("mem_contents.dat", mem_array);
                // display contents of mem_array
                for (k = 1; k <= 1024; k = k + 1)
                        $display("word[%d]=%h", k, 
                                mem_array[k]);
        end
endmodule
\end{lstlisting}
%
%%%%%%%%%%%%%%%%%%%%%%%%%%%%%%%%%%%%%%%%%%%%%%%%%%
\section*{Concurrency and Race Conditions}\vspace{-10pt}
Indeterminate outcomes result from race conditions when multiple behaviors use the procedural assignment operator ($=$) to reference (read) and assign value to teh same variable at the same time.
\subsection*{Example (with race):}\vspace{-10pt}
\begin{lstlisting}
always @ (posedge clock) a = b;
always @ (posedge clock) b = a;
\end{lstlisting}
Use nonblocking assignment operator ($<=$) to eliminate race conditions; assignments are independent of the order of the behaviors and the order of the statements.
\subsection*{Example:}\vspace{-10pt}
\begin{lstlisting}
always @ (posedge clock) a <= b;
always @ (posedge clock) b <= a;
\end{lstlisting}
Use the procedural assignment operator ($=$) in level-sensitive behaviors and the nonblocking assignment operator ($<=$) in edge-sensitive behaviors to avoid race conditions in sequential machines.
\subsection*{Example (without race):}\vspace{-10pt}
\begin{lstlisting}
always @ (posedge clock) 
        state <= next_state;
always @ (state or inputs)
        case(state)
                state_0: begin next_state = state_5; ...
                end
        endcase
\end{lstlisting}
%
%%%%%%%%%%%%%%%%%%%%%%%%%%%%%%%%%%%%%%%%%%%%%%%%%%
\section*{Functions}\vspace{-10pt}
May invoke another function; may not invoke a task. Executes with zero delay time.\\
May not contain delay control (\#), event control (@), or \textbf{wait}.\\
Must have at least one input argument.\\
May not have \textbf{output} or \textbf{input} arguments.\\
Function name serves as a placeholder for a single returned value.
\subsection*{Example:}\vspace{-10pt}
\begin{lstlisting}
value = adder(a,b);
...
function [4:0] adder;
        input [3:0] a,b;
        adder = a + b;
endfunction
\end{lstlisting}
%
%%%%%%%%%%%%%%%%%%%%%%%%%%%%%%%%%%%%%%%%%%%%%%%%%%
\section*{Tasks}\vspace{-10pt}
May invoke other tasks and other functions.\\
May contain delay control(\#), event control (@), and \textbf{wait}.\\
May have zero or more arguments having mode \textbf{input, output, inout}.\\
Passes values by its arguments.
\subsection*{Example:}\vspace{-10pt}
\begin{lstlisting}
adder (sum, a, b);
...
task adder;
        output [4:0] sum;
        input [3:0] a, b;
        sum = a + b;
endtask
\end{lstlisting}
%
%%%%%%%%%%%%%%%%%%%%%%%%%%%%%%%%%%%%%%%%%%%%%%%%%%
\section*{Selected Compiler Directives}\vspace{-10pt}
\begin{lstlisting}
`define width = 16;
`include .../project_header.v
`timescale = 100ns/1ns  // time_units / precision
`ifdef ... `else ... `endif
\end{lstlisting}
\subsection*{Example:}\vspace{-10pt}
\begin{lstlisting}
module or_model (y, a, b);
        output y;
        input a, b;
        `ifdef cont_assign 
                // uses continuous assignment
                assign y = a | b;
        `else
                or G1 (y, a, b); // uses primitive
        `endif
endmodule       
\end{lstlisting}
%
%%%%%%%%%%%%%%%%%%%%%%%%%%%%%%%%%%%%%%%%%%%%%%%%%%
\section*{Simulation Output}\vspace{-10pt}
\begin{lstlisting}
$display ("string_of_info %d", variable);
integer K;
initial K = $fopen ("output_file");
always @ (event_control expression) // dump data
begin
        $fdisplay (K, "%h"', data[7:0]);
        ...
        $fclose ("output_file");
        $monitor ($time, "out_1=%b,out_2=%b",out_1,out_2);
        $monitor (K,"some_value=%h",address[15:0]);
        $monitoron;
        $monitoroff;
end
\end{lstlisting}
%
%%%%%%%%%%%%%%%%%%%%%%%%%%%%%%%%%%%%%%%%%%%%%%%%%%
\section*{Simulation Data Control}\vspace{-10pt}
\begin{lstlisting}
initial begin
        // dump simulation data into my_data.dump:
        $dumpfile("my_data.dump");
        // dump all signals:
        $dumpvars;
        // dump variables in module top:
        $dumpvars(1,top);
        // dump variables 2 levels below top.mod1:
        $dumpvars(2,top.mod1);
        // stop dump after 1000 time units:
        #1000 $dumpoff;
        // start/restart dump after 500 time units:
        #500 $dumpon;
        // suspend simulation:
        $stop;
        // terminate simulation after 1000 time units:
        #1000 $finish;
end
\end{lstlisting}
%
%%%%%%%%%%%%%%%%%%%%%%%%%%%%%%%%%%%%%%%%%%%%%%%%%%
\section*{Parameter Redefinition}
\subsection*{In-line (instance-based)}\vspace{-10pt}
\begin{lstlisting}
module Something();
        parameter size = 4;
        parameter width = 8;
        ...
endmodule
...
Something #(8,32) M1();
...
\end{lstlisting}\vspace{-10pt}
\subsection*{Indirect (hierarchical dereferencing):}\vspace{-10pt}
\begin{lstlisting}
module Annotate();
        ...
        defparam .Something.width = 16;
        ...
endmodule
\end{lstlisting}
%
%%%%%%%%%%%%%%%%%%%%%%%%%%%%%%%%%%%%%%%%%%%%%%%%%%
\section*{Constructs to Avoid in Synthesis}\vspace{-10pt}
\begin{itemize} \itemsep0pt \parskip0pt \parsep0pt
        \item           \textbf{time, real, realtime} variables
        \item           named \textbf{event}
        \item           \textbf{triand, trior, tri0, tri1} nets
        \item           vector ranges for integers
        \item           single-pass (\textbf{initial}) behavior
        \item           \textbf{assign ... deassign} procedural continuous assignment
        \item           \textbf{force ... release} procedural continuous assignment
        \item           \textbf{fork ... join} block (parallel activity flow)
        \item           \textbf{defparam} parameter substitution
        \item           Operators: Modulus(\textbf{\%}) and Division(\textbf{/}), except for division by 2%, or modulus 2
        \item           ===, !===
        \item           Primitives: \textbf{pullup, pulldown, tranif0, tranif1, rtran, rtranif0, rtranif1}
        \item           Hierarchical pathnames
        \item           Compiler directives
\end{itemize}
%
%%%%%%%%%%%%%%%%%%%%%%%%%%%%%%%%%%%%%%%%%%%%%%%%%%
%\begin{center}
\line(1,0){250}\\
%\end{center}
This material is used while creating the current document:
\begin{enumerate} \itemsep0pt \parskip0pt \parsep0pt
        \item   \textbf{IEEE 1364 Verilog-2001 Standard description}.
        \item   \textbf{"Advanced Digital Design with the Verilog HDL" by Michael D. Ciletti}
\end{enumerate}

\end{document}

